Un insieme può contenere un insieme vuoto "simbolo insieme di inseme vuoto" != "insieme vuoto".

La rappresentazione intensionale permette di definire un insieme specificandone una proprietà
caratteristica che distingue gli elementi


L’unione di due insiemi A e B si denota A ∪ B
è definita come A ∪ B = {x | x ∈ A ∨ x ∈ B} 
cioè, A ∪ B è l’insieme di tutti gli elementi che appartengono ad A oppure a B 


L’intersezione di due insiemi A e B si denota: A ∩ B
è definita come:
A ∩ B = {x | x ∈ A ∧ x ∈ B}
Cioè, A ∩ B è l’insieme di tutti gli elementi che appartengono sia ad A, sia a B

La differenza tra A e B si denota con: A \ B
ed è definita come:
A \ B = { x | x ∈ A ∧ x ∉ B }
Cioè, A \ B è l’insieme di tutti gli elementi che appartengono ad A ma non a B 
cioè tolgo l’intersezione 


La differenza simmetrica tra A e B è:
A△B = (A \ B) ∪ (B \ A) (quindi l'unione tra A e B tolta la loro intersezione)

Quando abbiamo un insieme i cui elementi sono tutti insiemi, lo chiamiamo una
famiglia di insiemi 
