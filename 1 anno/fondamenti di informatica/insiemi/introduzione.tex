\documentclass[a4paper,12pt]{article}
\usepackage[utf8]{inputenc}
\usepackage{amsmath,amssymb}

\begin{document}

\section*{Insiemi Numerici e Cardinalità}

\subsection*{Numeri Naturali (\( \mathbb{N} \))}
I numeri naturali sono l'insieme di tutti i numeri interi positivi, incluso lo zero:
\[
\mathbb{N} = \{0, 1, 2, 3, \ldots, n, n+1, \ldots\}.
\]

\subsubsection*{Induzione}
L'induzione è il processo che verifica una proprietà su tutti gli elementi di \( \mathbb{N} \). Si basa sul seguente schema:
\[
P(0) \text{ è vera} \quad \text{e} \quad \left[P(n) \Rightarrow P(S(n)) \right],
\]
dove \( S(n) \) è il successore di \( n \). Questo dimostra che ogni elemento è più piccolo del suo successore.

\subsection*{Numeri Interi (\( \mathbb{Z} \))}
L'insieme dei numeri interi \( \mathbb{Z} \) comprende tutti i numeri naturali e i loro opposti:
\[
\mathbb{Z} = \{\ldots, -2, -1, 0, 1, 2, \ldots\}.
\]
L'insieme \( \mathbb{N} \) è un sottoinsieme proprio di \( \mathbb{Z} \), ovvero \( \mathbb{N} \subset \mathbb{Z} \), perché non contiene numeri negativi.

\subsubsection*{Valore Assoluto}
Il valore assoluto di un numero intero è il numero privo di segno, cioè:
\[
|x| =
\begin{cases} 
x & \text{se } x \geq 0, \\ 
-x & \text{se } x < 0.
\end{cases}
\]

\subsection*{Numeri Razionali (\( \mathbb{Q} \))}
I numeri razionali sono espressi come il rapporto \( \frac{m}{n} \), dove \( m, n \in \mathbb{Z} \) e \( n \neq 0 \). Si indica con:
\[
\mathbb{Q} = \left\{\frac{m}{n} \; \middle| \; m \in \mathbb{Z}, \; n \in \mathbb{Z} \setminus \{0\} \right\}.
\]
Anche in questo caso, \( \mathbb{Z} \) è un sottoinsieme proprio di \( \mathbb{Q} \), ovvero \( \mathbb{Z} \subset \mathbb{Q} \).

\subsection*{Numeri Irrazionali (\( \mathbb{I} \))}
I numeri irrazionali sono numeri che non possono essere espressi come una frazione \( \frac{m}{n} \). Alcuni esempi sono:
\[
\pi, \; e, \; \sqrt{2}.
\]

\subsection*{Numeri Reali (\( \mathbb{R} \))}
L'insieme dei numeri reali è dato dall'unione dei numeri razionali e irrazionali:
\[
\mathbb{R} = \mathbb{Q} \cup \mathbb{I}.
\]

\subsection*{Sistemi Discreti e Continui}
Un sistema è:
\begin{itemize}
    \item \textbf{Discreto:} se costituito da elementi isolati. Un insieme discreto ha una cardinalità numerabile.
    \item \textbf{Continuo:} se non presenta spazi vuoti tra i suoi elementi.
\end{itemize}

\subsection*{Cardinalità}
La cardinalità \( |A| \) di un insieme \( A \) rappresenta il numero dei suoi elementi. 
\begin{itemize}
    \item L'insieme \( \mathbb{N} \) è numerabile perché possiamo associare ogni numero naturale a un elemento univoco.
    \item Anche \( \mathbb{Q} \) è numerabile. Questo può essere dimostrato attraverso un piano cartesiano che ordina le coppie \( (m, n) \).
    \item L'insieme vuoto (\( \varnothing \)) non ha elementi, quindi \( |\varnothing| = 0 \).
\end{itemize}

\subsection*{Rappresentazione degli Insiemi}
Esistono due rappresentazioni principali per descrivere un insieme:
\begin{enumerate}
    \item \textbf{Estensionale:} si elencano esplicitamente tutti gli elementi dell'insieme. Ad esempio:
    \[
    A = \{1, 2, 3\}.
    \]
    L'ordine degli elementi è irrilevante.
    \item \textbf{Intensionale:} si specificano le proprietà che caratterizzano gli elementi dell'insieme. Ad esempio:
    \[
    B = \{x \in \mathbb{N} \; | \; x \text{ è pari}\}.
    \]
\end{enumerate}

\end{document}
