i numeri naturali si indicano con N e sono tutti i numeri interi positivi, N = {0,1,2,3 .... n, n+1}

l'induzione è il processo che definisce che una prorpietà generica in tutto N <--> è vera
in 0 e se vera in n e vera anche in S(n), questa dimostra che un elemento è sempre più
piccolo del successore. 

l'insieme Z sono tutti i numeri interi, n è sottoinsieme proprio di Z, N C Z, 

Il valore assoluto di un numero intero è il numero privo di segno, (il suo opposto)

i numeri razionali sono i rapporti m/n tra due numeri interi, si indica con Q, Z C Q

Un sottoinsieme proprio C quando nel sottoinsieme non è incluso almeno un elemento dell'insieme

I numeri irrazionali sono numeri non esprimibili da frazioni EG: e o pigreco e si indica con I

L'insieme dei reali si indica con R ed è l'unione U tra Q ed I

un sistema `e discreto se costituito da elementi isolati e continuo se non vi sono spazi vuoti
In matematica, discreto si bassa sul concetto di cardinalità Un insieme `e discreto se (e solo se) i suoi elementi si
possono numerare

La cardinalità è il numero degli elementi in un insieme, dato un insieme A si dice la card
|A|. l'insieme degli N è numerabile perchè a 1 posso associare 1 ecc.... , anche gli insiemi
finiti sono numerabili, ANCHE I NUMERI RAZIONALI Q SONO NUMERABILI tramite un piano cartesiano

l'insieme senza elementi si rappresenta con "Simbolo insieme vuoto"

Ci sono due rappresentazioni, la rappresentazione estensionale consiste nell'elencare espli
citamente ogni elemento dell'insieme, l'ordine è irrilevante.
