\documentclass[a4paper,12pt]{article}
\usepackage{amsmath}
\usepackage{amssymb}
\usepackage{geometry}
\geometry{margin=1in}

\begin{document}

\section*{Concetto Grafico}

Considerando un punto $P_0$ nel grafico di una funzione e il coefficiente angolare di una retta tangente, consideriamo un suo intorno $x_0 - \delta$ e $x_0 + \delta$. In questo intorno, se la funzione valutata in un qualsiasi punto $c$ piccolo ha ordinata maggiore rispetto al corrispettivo considerato sulla retta tangente ("la funzione va verso l'alto"), allora la funzione è 	extbf{convessa}.

Se si verifica il caso contrario ("la funzione va verso il basso"), allora la funzione è 	extbf{concava}.

Si verifica un 	extbf{punto di flesso} se in un intorno di $x_0$ la funzione è convessa a sinistra e concava a destra.

\section*{Definizione Formale}

Considerando $f:(a,b)$ derivabile in $(a,b)$ e $x_0 \in (a,b)$:

\begin{enumerate}
\item Se $f''(x_0) > 0$, la funzione è 	convessa.
\item Se $f''(x_0) < 0$, la funzione è 	concava.
\item Se $f''(x_0) = 0$ e $f'''(x_0) \neq 0$, allora $x_0$ è un 	punto di flesso.
\end{enumerate}

\section*{Esempio}

Consideriamo $f(x) = e^x - e^{-x} + 1$. Studiamo il comportamento in $x = -1$, $x = 0$, e $x = 1$:

\begin{itemize}
\item Derivata prima: 
\item Derivata seconda: 
\end{itemize}

\subsection*{Calcoli}\begin{itemize}
\item $f''(-1) = e^{-1} - e^1 = \frac{1}{e} - e < 0$ \quad \textbf{Punto di concavità in} $x=-1$.
\item $f''(1) = e^1 - e^{-1} = e - \frac{1}{e} > 0$ \quad \textbf{Punto di convessità in} $x=1$.
\item $f''(0) = e^0 - e^0 = 0$, quindi calcoliamo $f'''(x) = e^x + e^{-x}$:

\textbf{Punto di flesso in} $x=0$.
\end{itemize}

\end{document}
