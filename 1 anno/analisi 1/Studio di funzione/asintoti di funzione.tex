\documentclass{article}
\usepackage{amsmath}
\usepackage{amssymb}

\begin{document}

\section*{Asintoti Orizzontali, Verticali e Obliqui}

\subsection*{Asintoti Orizzontali}

Abbiamo un asintoto orizzontale quando, considerando una retta orizzontale, la distanza tra essa e il grafico della funzione tende a 0.  

Nella pratica, se 

\[
\lim_{x \to +\infty} f(x) = k \quad \text{con} \quad k \in \mathbb{R},
\]

allora ci potrebbe essere un asintoto orizzontale. Ma se

\[
\lim_{x \to -\infty} f(x) = h \quad \text{con} \quad h \in \mathbb{R} \quad \text{e} \quad h \neq k,
\]

la funzione ammette due asintoti orizzontali.

\subsection*{Asintoti Verticali}

Lo stesso vale per l'asintoto verticale. Considerando una retta verticale, nella pratica si ha che

\[
\lim_{x \to c} f(x) = +\infty \quad \text{o} \quad \lim_{x \to c} f(x) = -\infty.
\]

\subsection*{Esempio con Asintoto Verticale}

Consideriamo la funzione

\[
f(x) = \frac{x}{|x - 2|}, \quad D = (-\infty, 2) \cup (2, +\infty),
\]

con il punto di discontinuità in \( x = 2 \). Calcoliamo i limiti a destra e a sinistra di 2:

\[
\lim_{x \to 2^+} f(x) = \frac{2}{0^+} = +\infty.
\]

In questo caso, la funzione ha un asintoto verticale in \( x = 2 \).

\subsection*{Esempio con Asintoto Obliquo}

Consideriamo la funzione

\[
f(x) = \frac{x |x| + x^2 + 1}{x}, \quad D = (-\infty, 0) \cup (0, +\infty).
\]

Verifichiamo prima la presenza di un asintoto verticale. Calcoliamo i limiti a destra e a sinistra di 0:

\[
\lim_{x \to 0^+} f(x) = +\infty,
\]
\[
\lim_{x \to 0^-} f(x) = -\infty.
\]

Poiché i limiti a destra e a sinistra sono opposti e tendono all'infinito, \( x = 0 \) è un asintoto verticale.

\subsection*{Asintoti Orizzontali}

Verifichiamo la presenza di un asintoto orizzontale. Calcoliamo il limite per \( x \to -\infty \):

\[
\lim_{x \to -\infty} f(x) = \lim_{x \to -\infty} \frac{2x^2 + 1}{x} = 0,
\]

quindi \( y = 0 \) è un asintoto orizzontale.

\subsection*{Asintoto Obliquo}

Infine, verifichiamo l'esistenza di un asintoto obliquo. Calcoliamo il limite di \( \frac{f(x)}{x} \) per \( x \to +\infty \):

\[
m = \lim_{x \to +\infty} \frac{f(x)}{x} = \lim_{x \to +\infty} \frac{2x^2 + 1}{x^2} = 2.
\]

Poiché \( m = 2 \neq 0 \) e \( m \neq \infty \), possiamo affermare che esiste un asintoto obliquo. Calcoliamo il termine complementare \( q \):

\[
q = \lim_{x \to +\infty} \left( f(x) - mx \right) = \lim_{x \to +\infty} \left( \frac{2x^2 + 1}{x} - 2x \right) = 0.
\]

Poiché \( q = 0 \), la funzione ammette un asintoto obliquo di equazione \( y = 2x \).

\end{document}
