\documentclass{article}
\usepackage[utf8]{inputenc}
\usepackage{amsmath}
\usepackage{enumitem}

\begin{document}

\begin{Large}

Una serie a termini alterni si presenta come una successione a termini alterni, per esempio:

\[
\sum_{n=1}^{\infty} (-1)^{n+1} \cdot \frac{1}{n}
\]

questa risulta \(1\), \(-\frac{1}{2}\), \(+\frac{1}{3}\), \(-\frac{1}{4}\), \(+\frac{1}{5}\), \(\ldots\)

Per questi tipi di serie si applica il criterio di Leibniz, che dice che una serie a termini alterni

\[
\sum_{n=0}^{\infty} (-1)^{n} \cdot A_n
\]

la serie converge SE:

\begin{enumerate}
    \item Il limite per \(n\) che tende a infinito del termine \(A_n\) è 0 (infinitesimo).
    \item La successione \(|A_1|\), \(|A_2|\), \(|A_3|\), \(\ldots\), \(|A_n|\) DEVE ESSERE DECRESCENTE.
\end{enumerate}

\end{Large}

\end{document}
