\documentclass[12pt]{article}  % Imposta una dimensione del carattere più grande
\usepackage{amsmath}           % Pacchetto per formule matematiche avanzate
\usepackage{amsfonts}          % Pacchetto per i font matematici
\usepackage{geometry}          % Per gestire i margini del documento
\geometry{a4paper, margin=1in} % Margini per migliorare la leggibilità

\begin{document}

\section*{Criterio del Rapporto, Criterio della Radice e Proprietà Utili}

\subsection*{Proprietà Utili}

\begin{enumerate}
    \item Se abbiamo $\sum_{n=0}^{\infty} (A_n + B_n)$, possiamo riscriverla come la somma delle singole serie:
    \[
    \sum_{n=0}^{\infty} A_n + \sum_{n=0}^{\infty} B_n,
    \]
    dove $A_n$ e $B_n$ sono due serie numeriche convergenti.

    \item Se abbiamo $\sum_{n=0}^{\infty} K A_n$, dove $K$ è una costante, si può riscrivere come
    \[
    K \cdot \sum_{n=0}^{\infty} A_n.
    \]

    \item \textbf{Importante}: Una serie $\sum_{n=0}^{\infty} A_n$ a termini non negativi, con $A_n \geq 0$, può convergere a $0$ o divergere a $+\infty$. \textbf{Non può essere indeterminata}.
\end{enumerate}

Un esempio è la serie $\sum_{n=0}^{\infty} \frac{n^2 + 3^{-n}}{n^2 - 5}$, che non può convergere poiché $A_n$ tende a $1$ per $n \to \infty$. Perché converga, il limite di $A_n$ dovrebbe essere $0$.

\subsection*{Criterio del Rapporto}

Sia $A_n > 0$ definitivamente. Consideriamo il limite
\[
\lim_{n \to \infty} \frac{A_{n+1}}{A_n} = L \in [0, +\infty).
\]
Allora, se:
\begin{enumerate}
    \item $L < 1$, la serie $\sum_{n=0}^{\infty} A_n$ converge.
    \item $L > 1$, la serie $\sum_{n=0}^{\infty} A_n$ diverge.
    \item $L = 1$, il criterio non fornisce informazioni, quindi è necessario usare un altro criterio.
\end{enumerate}

\subsection*{Criterio della Radice}

Sia $A_n \geq 0$ definitivamente. Consideriamo il limite
\[
\lim_{n \to \infty} \sqrt[n]{A_n} = L.
\]
Allora, se:
\begin{enumerate}
    \item $L < 1$, la serie $\sum_{n=0}^{\infty} A_n$ converge.
    \item $L > 1$, la serie $\sum_{n=0}^{\infty} A_n$ diverge.
    \item $L = 1$, il criterio non fornisce informazioni, quindi è necessario usare un altro criterio.
\end{enumerate}

Questo criterio consiste nel prendere la radice $n$-esima del termine generale $A_n$.

\subsection*{Esempio}

Consideriamo la serie
\[
\sum_{n=2}^{\infty} \frac{1}{\log(n)^{\frac{n}{2}}}.
\]
Applicando il criterio della radice:
\[
\lim_{n \to \infty} \left( \log(n)^{-\frac{n}{2}} \right)^{\frac{1}{n}} = \lim_{n \to \infty} \log(n)^{-\frac{1}{2}} = \lim_{n \to \infty} \frac{1}{\sqrt{\log(n)}} = 0 < 1.
\]
Quindi la serie converge.

\end{document}
