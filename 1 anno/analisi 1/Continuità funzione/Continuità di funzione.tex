\documentclass[12pt]{article}
\usepackage{amsmath}
\usepackage{amssymb}
\usepackage{geometry}

\geometry{a4paper, margin=1in}

\begin{document}
\Huge

\textbf{Continuità di una funzione:}  

Consideriamo una funzione \( f(x): X \to Y \).  
Nel punto \( x_0 \), \( f(x) \) è continua quando si verificano le seguenti tre condizioni:  

\begin{enumerate}
    \item \( f(x_0) \) deve esistere, e quindi \( x_0 \) deve appartenere al dominio della funzione.
    \item Esiste \textbf{finito} il limite:  
    \[
    \lim_{x \to x_0} f(x) = L
    \]
    \item Deve essere verificata l'uguaglianza:  
    \[
    f(x_0) = L
    \]
\end{enumerate}

\vspace{0.5cm}

\textbf{Esempio:}  

Sia \( f(x) = x^2 + |x - 1| \).  
Verifichiamo se la funzione è continua nel punto \( x = 1 \):  

\begin{itemize}
    \item Calcoliamo \( f(1) \):  
    \[
    f(1) = 1^2 + |1 - 1| = 1 + 0 = 1
    \]

    \item Calcoliamo il limite:  
    \[
    \lim_{x \to 1} \big(x^2 + |x - 1|\big) = 1
    \]

    \item Verifichiamo che \( f(1) = \lim_{x \to 1} f(x) \):  
    \[
    f(1) = 1 \quad \text{e} \quad \lim_{x \to 1} f(x) = 1
    \]
    Entrambi coincidono.
\end{itemize}

\vspace{0.5cm}

Quindi, la funzione \( f(x) \) è continua nel punto \( x = 1 \).

\end{document}
