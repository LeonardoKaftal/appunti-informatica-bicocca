\documentclass[12pt]{article}
\usepackage{amsmath}
\usepackage{amssymb}
\usepackage{geometry}

\geometry{a4paper, margin=1in}

\begin{document}
\Huge

\section*{Teorema di Rolle}

Consideriamo una funzione \( f(x) \) continua su un intervallo chiuso \([a, b]\) e derivabile nei punti interni di \((a, b)\) (escludendo quindi i punti della frontiera).  

Se \( f(a) = f(b) \), allora esiste un punto \( c \in (a, b) \) (punto interno) tale che:  
\[
f'(c) = 0
\]

Inoltre, per il Teorema di Weierstrass, esisteranno anche il minimo e il massimo assoluto della funzione sull’intervallo.  

\vspace{0.5cm}

\textbf{Conclusioni:}
\begin{itemize}
    \item Esiste un minimo \( m \) e un massimo \( M \) tali che:  
    \[
    m \leq f(x) \leq M \quad \forall x \in [a, b]
    \]
    \item Se \( m < M \), almeno uno tra il minimo o il massimo assoluto si trova in un punto interno dell’intervallo.
    \item Se \( m = M \), la funzione è costante e la derivata prima è \( f'(x) = 0 \) su tutto l’intervallo.
\end{itemize}

\vspace{1cm}

\section*{Esempio:}

Sia \( f(x) = \ln(-x^2 - 2x + 3) \), e consideriamo l'intervallo \([-2, 0]\).  

\textbf{Passi per la verifica:}
\begin{enumerate}
    \item \textbf{Dominio:}  
    Il dominio della funzione è:  
    \[
    -x^2 - 2x + 3 > 0 \implies x \in (-3, 1)
    \]
    L'intervallo \([-2, 0]\) è contenuto nel dominio, quindi la funzione è continua su \([-2, 0]\).

    \item \textbf{Valori agli estremi:}  
    \[
    f(-2) = \ln(3) \quad \text{e} \quad f(0) = \ln(3)
    \]
    Poiché \( f(-2) = f(0) \), le ipotesi del teorema sono soddisfatte.

    \item \textbf{Derivata della funzione:}  
    \[
    f'(x) = \frac{-2x - 2}{-x^2 - 2x + 3}
    \]
    La funzione è derivabile in tutti i punti dell'intervallo \([-2, 0]\) (il denominatore non si annulla).

    \item \textbf{Ricerca del punto \( c \):}  
    Cerchiamo \( c \in (-2, 0) \) tale che \( f'(c) = 0 \):
    \[
    f'(c) = \frac{-2c - 2}{-c^2 - 2c + 3} = 0
    \]
    Da cui si ottiene:
    \[
    -2c - 2 = 0 \implies 2c = -2 \implies c = -1
    \]
    Il punto \( c = -1 \) appartiene all’intervallo \((-2, 0)\) ed è un punto interno.
\end{enumerate}

\vspace{0.5cm}

\textbf{Conclusione:}  
La funzione soddisfa le ipotesi del teorema, ed esiste il punto \( c = -1 \) in cui \( f'(c) = 0 \).

\end{document}
