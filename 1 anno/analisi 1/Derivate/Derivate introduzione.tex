\documentclass[12pt]{article}
\usepackage{amsmath}
\usepackage{amssymb}
\usepackage{geometry}
\usepackage{lipsum}

\geometry{a4paper, margin=1in}

\begin{document}
\Huge

Consideriamo di avere una funzione \( f(x): X \to Y \), dove la funzione è continua, e \( f(x_0) = x_0 \) (trasposto).

\vspace{0.5cm}

Consideriamo una variazione \( h \): 
\[
x_0 + h = y_1
\]
quindi la funzione è variata in \( y \).

\vspace{0.5cm}

\textbf{IMPORTANTE:} Il \textit{rapporto incrementale} è definito come: 
\[
\frac{\Delta y}{\Delta x} = \frac{f(x_0 + h) - f(x_0)}{h}
\]

Deve esistere il limite:
\[
\lim_{h \to 0} \frac{f(x_0 + h) - f(x_0)}{h}
\]

\vspace{0.5cm}

Per esempio, consideriamo:
\[
\lim_{h \to 0} \frac{f(x + h) - f(x)}{h} = f'(x)
\]

Se questo limite esiste, allora la derivata prima è continua. Per esempio:

\vspace{0.5cm}

Quanto vale la derivata di \( x \)? Calcoliamo il rapporto incrementale:
\[
f(x + h) = \frac{\lim_{h \to 0} (x + h) - x}{h} = \frac{h}{h} = 1
\]

Quindi:
\[
f'(x) = 1
\]

Quindi riassumendo, la derivata prima è il coefficiente angolare della retta tangente nel
punto x.

\end{document}
