\documentclass[12pt]{article}
\usepackage{amsmath}
\usepackage{amssymb}
\usepackage{geometry}

\geometry{a4paper, margin=1in}

\begin{document}
\Huge

\textbf{Regole di derivazione:}  

\vspace{0.5cm}

1. \textbf{Somma di funzioni:}  
Se \( f(x) = a(x) + b(x) + c(x) \), allora la derivata si calcola derivando ogni termine:  
\[
f'(x) = a'(x) + b'(x) + c'(x)
\]

\vspace{0.5cm}

2. \textbf{Prodotto di funzioni:}  
Se \( f(x) = a(x) \cdot b(x) \cdot c(x) \), allora la derivata è:  
\[
f'(x) = a'(x) \cdot b(x) \cdot c(x) + b'(x) \cdot a(x) \cdot c(x) + c'(x) \cdot a(x) \cdot b(x)
\]

\vspace{0.5cm}

3. \textbf{Rapporto di funzioni:}  
Se \( f(x) = \frac{a(x)}{b(x)} \), allora la derivata è:  
\[
f'(x) = \frac{a'(x) \cdot b(x) - b'(x) \cdot a(x)}{b(x)^2}
\]

\vspace{0.5cm}

\textbf{Funzioni composte:}  

\vspace{0.5cm}

\begin{enumerate}
    \item Se \( f(x) = \big(g(x)\big)^a \), allora:  
    \[
    f'(x) = a \cdot \big(g(x)\big)^{a - 1} \cdot g'(x)
    \]

    \vspace{0.5cm}

    \item Se \( f(x) = a^{g(x)} \), allora:  
    \[
    f'(x) = a^{g(x)} \cdot \ln(a) \cdot g'(x)
    \]

    \vspace{0.5cm}

    \item Se \( f(x) = \ln(g(x)) \), allora:  
    \[
    f'(x) = \frac{1}{g(x)} \cdot g'(x)
    \]
\end{enumerate}

\end{document}
