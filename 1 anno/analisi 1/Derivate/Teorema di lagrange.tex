\documentclass[12pt]{article}
\usepackage{amsmath}
\usepackage{amssymb}
\usepackage{geometry}
\geometry{a4paper, margin=1in}

\begin{document}
\Huge

\section*{Teorema di Lagrange}

Sia \( f(x) \) una funzione continua su un intervallo chiuso \([a, b]\) e derivabile in \((a, b)\).  
Allora, \textbf{esiste un punto interno} \( c \in (a, b) \) tale che:
\[
f'(c) = \frac{f(b) - f(a)}{b - a}
\]
(Il teorema è simile a quello di Rolle, con la differenza che il risultato non è necessariamente nullo.)

\vspace{0.8cm}

\section*{Dimostrazione}

Consideriamo una funzione ausiliaria \( \phi(x) \), definita come la somma di \( f(x) \) con un termine lineare \( kx \):
\[
\phi(x) = f(x) + kx
\]
Essendo \( f(x) \) continua e derivabile per ipotesi, e \( kx \) una funzione lineare (quindi anch'essa continua e derivabile), segue che \( \phi(x) \) è continua su \([a, b]\) e derivabile in \((a, b)\).

Calcoliamo la derivata:
\[
\phi'(x) = f'(x) + k
\]
Ora imponiamo che \( \phi(x) \) sia uguale nei due estremi dell'intervallo:
\[
\phi(a) = \phi(b)
\]
Quindi:
\[
f(a) + ka = f(b) + kb
\]
da cui segue:
\[
f(b) - f(a) = -k(b - a)
\]
Ricavando \( k \):
\[
k = -\frac{f(b) - f(a)}{b - a}
\]

Per il teorema di Rolle, esiste \( c \in (a, b) \) tale che:
\[
\phi'(c) = 0
\]
cioè:
\[
f'(c) + k = 0
\]
Sostituendo \( k \):
\[
f'(c) = -k = \frac{f(b) - f(a)}{b - a}
\]
Questo conclude la dimostrazione.

\vspace{0.8cm}

\section*{Esempio}

\begin{enumerate}
    \item Consideriamo \( f(x) = \arcsin(x) \) su \([-1, 1]\).

    Calcoliamo la derivata:
    \[
    f'(x) = \frac{1}{\sqrt{1 - x^2}}
    \]
    Per il teorema, esiste un \( c \in (-1, 1) \) tale che:
    \[
    f'(c) = \frac{f(1) - f(-1)}{1 - (-1)}
    \]
    Calcoliamo gli estremi:
    \[
    f(1) = \frac{\pi}{2}, \quad f(-1) = -\frac{\pi}{2}
    \]
    Quindi:
    \[
    f'(c) = \frac{\frac{\pi}{2} - \left(-\frac{\pi}{2}\right)}{2} = \frac{\pi}{2}
    \]
    Sostituendo nella derivata:
    \[
    \frac{1}{\sqrt{1 - c^2}} = \frac{\pi}{2}
    \]
    Risolvendo:
    \[
    \sqrt{1 - c^2} = \frac{2}{\pi}, \quad 1 - c^2 = \frac{4}{\pi^2}
    \]
    Da cui:
    \[
    c^2 = 1 - \frac{4}{\pi^2}
    \]
    Le soluzioni sono:
    \[
    c_1 = -\sqrt{1 - \frac{4}{\pi^2}}, \quad c_2 = \sqrt{1 - \frac{4}{\pi^2}}
    \]
\end{enumerate}

\end{document}
