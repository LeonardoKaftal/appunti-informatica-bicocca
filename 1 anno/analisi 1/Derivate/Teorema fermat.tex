\documentclass[12pt]{article}
\usepackage{amsmath}
\usepackage{amssymb}
\usepackage{geometry}

\geometry{a4paper, margin=1in}

\begin{document}
\Huge

\section*{Teorema di Fermat}

Sia \( f(x) \) una funzione continua e limitata su un intervallo chiuso \([a, b]\), e derivabile nei punti interni \((a, b)\) (esclusi i punti di frontiera).  

\vspace{0.5cm}

\textbf{Condizioni necessarie:}
\begin{enumerate}
    \item La funzione \( f(x) \) deve essere \textbf{continua e limitata} su \([a, b]\).
    \item Deve esistere almeno un punto di \textbf{massimo} o di \textbf{minimo} assoluto.  
    \item Il punto di massimo o minimo, denotato come \( x_0 \), deve essere interno all'intervallo \((a, b)\), ovvero \( x_0 \notin \{a, b\}\).
    \item La funzione deve essere derivabile nel punto \( x_0 \).
\end{enumerate}

\vspace{0.5cm}

\textbf{Conclusione:}  
Nel punto di massimo o minimo interno \( x_0 \), la derivata prima si annulla:
\[
f'(x_0) = 0
\]

\vspace{1cm}

\section*{Osservazioni:}

\begin{itemize}
    \item Se la funzione \( f(x) \) ha un massimo o minimo assoluto su un intervallo, deve verificarsi almeno in un punto. 
    \item Tale punto può essere interno (derivabile, con \( f'(x_0) = 0 \)) oppure sulla frontiera (non richiede derivabilità).
    \item Il teorema richiede che il massimo e il minimo considerati siano interni per garantire che \( f'(x_0) = 0 \).
\end{itemize}
\end{document}
