\documentclass[12pt]{article}
\usepackage{amsmath}
\usepackage{amssymb}
\usepackage{geometry}

\geometry{a4paper, margin=1in}

\begin{document}
\Huge

\textbf{Punti di non derivabilità:}  
I punti di non derivabilità includono:  
\begin{itemize}
    \item Punto angoloso
    \item Cuspide
    \item Flesso a tangente verticale
\end{itemize}

\vspace{0.5cm}

\textbf{Punto angoloso:}  
Consideriamo un punto \( x_0 \) in cui la funzione è continua.  
C'è un punto angoloso quando:  
\[
\lim_{x \to x_0^-} f'(x) = m_1 \quad \text{(numero finito)} \quad \text{ed esiste anche} \quad \lim_{x \to x_0^+} f'(x) = m_2
\]
con \( m_1 \neq m_2 \).

Per esempio, una funzione con punto angoloso è:  
\[
f(x) = 
\begin{cases} 
-\arctan(x) & \text{se } x < 0 \\ 
\arctan(x) & \text{se } x \geq 0
\end{cases}
\]

Per \( x < 0 \), la derivata è:  
\[
-\frac{1}{1 + x^2}
\]
Per \( x > 0 \), la derivata è:  
\[
\frac{1}{1 + x^2}
\]

Calcoliamo i limiti:  
\[
\lim_{x \to 0^-} \left( -\frac{1}{1 + x^2} \right) = -1
\]
\[
\lim_{x \to 0^+} \left( \frac{1}{1 + x^2} \right) = 1
\]
I due limiti sono diversi, quindi il punto \( x = 0 \) è un punto angoloso.

\vspace{0.5cm}

\textbf{Punto di cuspide:}  
Si ha un punto di cuspide quando:  
\[
\lim_{x \to x_0^-} f'(x) = +\infty \quad \text{e} \quad \lim_{x \to x_0^+} f'(x) = -\infty
\]
(oppure il contrario, basta che siano opposti).

Per esempio, consideriamo:
\[
f(x) = \sqrt[3]{x^2} + 1
\]
La derivata è:
\[
f'(x) = \frac{2}{3} \cdot \frac{x}{\sqrt[3]{x^2}}
\]

Calcoliamo i limiti:  
\[
\lim_{x \to 0^-} f'(x) = -\infty \quad \text{e} \quad \lim_{x \to 0^+} f'(x) = +\infty
\]
Quindi \( x = 0 \) è un punto di cuspide.

\vspace{0.5cm}

\textbf{Flesso a tangente verticale:}  
Una funzione ha un flesso a tangente verticale di non derivabilità quando il limite da destra e da sinistra risulta \( +\infty \) o \( -\infty \).

Per esempio:
\[
f(x) = \sqrt[3]{x} + 1
\]
La derivata è:  
\[
f'(x) = \frac{1}{3} \cdot x^{-2/3} = \frac{1}{3} \cdot \frac{1}{\sqrt[3]{x^2}}
\]

Calcoliamo i limiti:
\[
\lim_{x \to 0^-} f'(x) = +\infty \quad \text{e} \quad \lim_{x \to 0^+} f'(x) = +\infty
\]
Quindi \( x = 0 \) è un flesso a tangente verticale.

\end{document}
